\documentclass[a4paper]{article}
\usepackage[
	top=1in,
	bottom=1.75in,
	left=1.25in,
	right=1.25in
]{geometry}
\usepackage{algorithm}
\usepackage{amsmath}
\usepackage{algpseudocode}
\usepackage{pdfpages}
\usepackage{amssymb}
\usepackage{qtree}
\newcommand\qlabelhook{\it}
\usepackage{graphicx}
\usepackage{caption}
\usepackage{subcaption}

\def\sizeablebox[#1,#2,#3]{
		\hbox to #2 {\framebox[#1]{#3}}%
	}
\def\qframesubtree{\setbox\treeboxone \hbox{\framebox[1\width]{\box\treeboxone}}}

\usepackage[UKenglish]{babel}% http://ctan.org/pkg/babel
\usepackage[UKenglish]{isodate}% http://ctan.org/pkg/isodate
\cleanlookdateon

\begin{document}
	\title{Mixed Week 5}
	\author{Christian Nassif-Haynes}
	\maketitle
	
	\section*{Tutorial}
	\begin{enumerate}
		\setcounter{enumi}{2}
		\item In the third rule of the attribute grammar below, subscripts have been used to distinguish between occurrences of like nonterminals.
		\begin{align*}
			E &\rightarrow N && [E.ones = N.ones] \\
			N &\rightarrow B && [N.ones = B.ones] \\
			N_1 &\rightarrow N_2 \ B && [N_1.ones = N_2.ones + B.ones] \\
			B &\rightarrow \textbf{0} && [B.ones = 0] \\
			B &\rightarrow \textbf{1} && [B.ones = 1]\\
		\end{align*}
		\item Note that $13_{10} = 1101_2$.
		
		\Tree [.{E.ones = 3}
				[.{N.ones = 3}
					[.{N.ones = 2}
						[.{N.ones = 2}
							[.{N.ones = 1}
								[.{B.ones = 1}
									[ 1 ]]]
							[.{B.ones = 1}
								[ 1 ]]]
						[.{B.ones = 0}
							[ 0 ]]]
					[.{B.ones = 1}
						[ 1 ]]]]
	\end{enumerate}
\end{document}
