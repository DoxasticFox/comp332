\documentclass[a4paper]{article}
\usepackage[
	top=1in,
	bottom=1.75in,
	left=1.25in,
	right=1.25in
]{geometry}
\usepackage{algorithm}
\usepackage{amsmath}
\usepackage{algpseudocode}
\usepackage{pdfpages}
\usepackage{amssymb}
\usepackage{qtree}
\newcommand\qlabelhook{\it}
\usepackage{graphicx}
\usepackage{caption}
\usepackage{subcaption}

\def\sizeablebox[#1,#2,#3]{
		\hbox to #2 {\framebox[#1]{#3}}%
	}
\def\qframesubtree{\setbox\treeboxone \hbox{\framebox[1\width]{\box\treeboxone}}}

\usepackage[UKenglish]{babel}% http://ctan.org/pkg/babel
\usepackage[UKenglish]{isodate}% http://ctan.org/pkg/isodate
\cleanlookdateon

\begin{document}
	\title{Workshop Week 4}
	\author{Christian Nassif-Haynes}
	\maketitle

	\begin{enumerate}
		\item The starting symbol of the below grammar is $Expr$. Perl-style regular expressions are used to describe some terminals and are enclosed in forward slashes.
		\begin{align}
			Expr			&\rightarrow ExprRightMid \;\;|\;\; ExprRightMid \;\; Op3 \;\; ExprRightMid \\
			ExprRightMid	&\rightarrow ExprLeftHi \;\;|\;\; ExprLeftHi \;\; Op2 \;\; ExprRightMid \\
			ExprLeftHi		&\rightarrow Oprand \;\;|\;\; ExprLeftHi \;\; Op1 \;\; Oprand  \\
			Oprand			&\rightarrow Int \;\;|\;\; Ident \;\;|\;\; \text{\bf ( } ExprRightMid \text{ \bf ) } \\
			Op1				&\rightarrow \text{\bf ?} \\
			Op2				&\rightarrow \text{\bf !} \;\;|\;\; \text{\bf \^\null} \\
			Op3				&\rightarrow \text{\bf @} \\
			Int				&\rightarrow \text{\bf/{[}0-9{]}+/} \\
			Ident			&\rightarrow \text{\bf/{[}a-zA-Z{]}+/}
		\end{align}
		\item The left recursive productions are used to allow for left associativity.
		\item The derivation of \verb|(a ^ 2) ? b| is as follows:
		\begin{align*}
			Expr &\rightarrow ExprRightMid & \text{Rule 1} \\
				 &\rightarrow ExprLeftHi & \text{Rule 2} \\
				 &\rightarrow ExprLeftHi \;\; Op1 \;\; Oprand & \text{Rule 3} \\
				 &\rightarrow Oprand \;\; Op1 \;\; Oprand & \text{Rule 3} \\
				 &\rightarrow \text{\bf ( } ExprRightMid \text{ \bf ) } \;\; Op1 \;\; Oprand & \text{Rule 4} \\
				 &\rightarrow \text{\bf ( } ExprLeftHi \;\; Op2 \;\; ExprRightMid \text{ \bf ) } \;\; Op1 \;\; Oprand & \text{Rule 2} \\
				 &\rightarrow \text{\bf ( } Oprand \;\; Op2 \;\; ExprRightMid \text{ \bf ) } \;\; Op1 \;\; Oprand & \text{Rule 3} \\
				 &\rightarrow \text{\bf ( } Ident \;\; Op2 \;\; ExprRightMid \text{ \bf ) } \;\; Op1 \;\; Oprand & \text{Rule 4} \\
				 &\rightarrow \text{\bf ( a} \;\; Op2 \;\; ExprRightMid \text{ \bf ) } \;\; Op1 \;\; Oprand & \text{Rule 9} \\
				 &\rightarrow \text{\bf ( a \^\null} \;\; ExprRightMid \text{ \bf ) } \;\; Op1 \;\; Oprand & \text{Rule 6} \\
				 &\rightarrow \text{\bf ( a \^\null} \;\; ExprLeftHi \text{ \bf ) } \;\; Op1 \;\; Oprand & \text{Rule 2} \\
				 &\rightarrow \text{\bf ( a \^\null} \;\; Oprand \text{ \bf ) } \;\; Op1 \;\; Oprand & \text{Rule 3} \\
				 &\rightarrow \text{\bf ( a \^\null} \;\; Int \text{ \bf ) } \;\; Op1 \;\; Oprand & \text{Rule 4} \\
				 &\rightarrow \text{\bf ( a \^\null} \;\; 2 \text{ \bf ) } \;\; Op1 \;\; Oprand & \text{Rule 8} \\
				 &\rightarrow \text{\bf ( a \^\null} \;\; 2 \text{ \bf ) ?} \;\; Oprand & \text{Rule 5} \\
				 &\rightarrow \text{\bf ( a \^\null} \;\; 2 \text{ \bf ) ?} \;\; Ident & \text{Rule 4} \\
				 &\rightarrow \text{\bf ( a \^\null} \;\; 2 \text{ \bf ) ? b} & \text{Rule 9} \\
		\end{align*}
		\item
			\Tree[.Expr
					[.ExprRightMid
						[.ExprLeftHi
							[.ExprLeftHi
								[.Oprand
									[ ( ]
									[.ExprRightMid
										[.ExprLeftHi
											[.Oprand
												[.Ident
													[ a ]]]]
										[.Op2
											[ \text{\^\null} ]]
										[.ExprRightMid
											[.ExprLeftHi
												[.Oprand
													[.Int
														[ 2 ]]]]]]
									[ ) ]]]
							[.Op1
								[ ? ]]
							[.Oprand
								[.Ident
									[ b ]]]]]]
	\end{enumerate}
\end{document}
