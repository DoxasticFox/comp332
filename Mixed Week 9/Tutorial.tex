\documentclass[a4paper]{article}
\usepackage[
	top=1in,
	bottom=1in,
	left=1in,
	right=1in
]{geometry}
\usepackage{amssymb}
\usepackage{amsmath}
\usepackage{MnSymbol}

\usepackage[UKenglish]{babel}% http://ctan.org/pkg/babel
\usepackage[UKenglish]{isodate}% http://ctan.org/pkg/isodate
\cleanlookdateon

\begin{document}
	\title{Mixed Week 9}
	\author{Christian Nassif-Haynes}
	\maketitle
	
	\section*{Tutorial}
	\begin{enumerate}
		\setcounter{enumi}{2}
		\item Compilation can by done by first assigning each variable a memory address:
		\begin{verbatim}
0xC0 <- 1;
0xC1 <- 100;
while (*0xC0) {
   0xC1 <- *0xC1 - 1
   if (0xC1 == 0)
       *0xC0 = 0;
}
		\end{verbatim}
		The code then can be rewritten in assembly:
		\begin{verbatim}
    lvalue      0xC0              ; int a = 1;
    push        1
    assign
    lvalue      0xC1              ; int b = 100;
    push        100
    assign
while_cond_000                    ; while loop
    rvalue      0xC0
    gofalse     while_end_000
while_start_000
    lvalue      0xC1              ; b-- is equivalent to b = b - 1
    rvalue      0xC1              ;     Evaluating b - 1...
    push        1                 ;     Evaluating b - 1...
    sub                           ;     Evaluating b - 1...
    assign
if_cond_000
    rvalue      0xC1              ; b == 0
    gofalse     if_end_000
if_start_000
    lvalue      0xC0              ; a = 0
    push        0
    assign
if_end_000
    goto        while_cond_000
while_end_000
		\end{verbatim}
		Labels for subsequent while loops and if statements should be numbered  (e.g. \verb=while_cond_001=).
	\end{enumerate}
\end{document}
