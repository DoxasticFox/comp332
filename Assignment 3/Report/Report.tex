\documentclass[a4paper]{article}
\usepackage[
	top=1in,
	bottom=1.75in,
	left=1.25in,
	right=1.25in
]{geometry}
\usepackage{algorithm}
\usepackage{amsmath}
\usepackage{algpseudocode}
\usepackage{pdfpages}
\usepackage{amssymb}
\usepackage{qtree}
\newcommand\qlabelhook{\it}
\usepackage{graphicx}
\usepackage{caption}
\usepackage{subcaption}
\usepackage{hyperref}

\def\sizeablebox[#1,#2,#3]{
		\hbox to #2 {\framebox[#1]{#3}}%
	}
\def\qframesubtree{\setbox\treeboxone \hbox{\framebox[1\width]{\box\treeboxone}}}

\usepackage[UKenglish]{babel}% http://ctan.org/pkg/babel
\usepackage[UKenglish]{isodate}% http://ctan.org/pkg/isodate
\cleanlookdateon

\begin{document}
	\begin{titlepage}
		\title{COMP332 -- Programming Languages \\[5pt]
		Assignment Three Report \vfill}
		\author{Christian Nassif-Haynes -- 42510023}
		\maketitle
	\end{titlepage}
	
	\section{Introduction}
	This document describes the implementation and testing of assignment two in COMP332. The aim of the assignment was to extend the given program to implement code translation for the Func332 functional programming language by using the Kiama library.\footnote{\url {https://code.google.com/p/kiama/}} The target language was SEC machine code.
	
	The following sections discuss the implementation of the translator and the way in which the project was tested.
	
	\section{The Translator}
	The SEC machine is a stack-based machine; computations are carried out by pushing and popping data onto and off of a stack. The operations performed on elements in the stack are dictated by low-level SEC instructions. These instructions, which comprise the program, occur in nested lists, together with other constructs (e.g. \verb=IClosure=, which takes variable names and lists of instructions as its arguments). The list of available instructions was taken from the assignment specification.
	
	Translation into SEC machine language occurred using two versions of a \verb=translate= function -- one accepting a \verb=Program= to be processed statement-by-statement, and another accepting \verb=Expression=s. The latter version translated the program recursively by adding instructions to a \verb=Frame= (which is really a kind of list). The precise instruction sequence to be generated by a given func332 statement was specified in the assignment.

	\section{Testing}
	The tests were written with the tree structures listed in \verb=Func332Tree= (e.g. \verb=PlusExp=, \verb=LetExp= and \verb=IntExp=) kept in mind. This is because the recursive step in translating the code does so using \verb=case= expressions which match these structures, and may be a source of error. As well, \verb=if=, \verb=let= and function expressions which are nested inside expressions of the same type were tested. Nesting of different types -- for instance in let expressions that bind functions -- was also tested, though these cases are not covered exhaustively.
\end{document}
